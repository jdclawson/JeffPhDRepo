\documentclass[11pt]{beamer}
\usetheme{Darmstadt}
\useoutertheme{split}
\useoutertheme{miniframes}
\usepackage{hyperref}
\usepackage{bm}
\usepackage{times}
\usefonttheme{structurebold}
\usepackage{array}
\usepackage[english]{babel}
\usepackage{amsmath,amssymb,amsthm,mathrsfs,amsfonts,dsfont}
\usepackage[latin1]{inputenc}
\usepackage{adjustbox}
\usepackage{graphicx}
%Code snippets and syntax highlighting
\usepackage{listings}
%Settings for Listings
\lstset{
  language=Python,
  basicstyle=\ttfamily,
  keywordstyle=\color{blue}\ttfamily,
  stringstyle=\color{red}\ttfamily,
  commentstyle=\color{green}\ttfamily,
  morecomment=[l][\color{magenta}]{\#}
}

\usepackage{array}
\usepackage{threeparttable}
\usepackage{booktabs}
\usepackage{colortbl}
\usepackage{multirow}
\usepackage{amsthm}
\usepackage{float,graphicx,color}
\newtheorem*{thm}{Theorem}
\theoremstyle{definition}
%\numberwithin{equation}{section}
\newtheorem*{defn}{Definition}
\newcommand\boldline{\arrayrulewidth{1pt}\hline}
\newcommand\ve{\varepsilon}


\setbeamercovered{dynamic}

\title[Econ 741 Project]{PAYG, Current Account and Fertility Rates: \\ Consequences for Savings}
\author[Clawson]{Jeff Clawson}
\date{\today}


\begin{document}

\begin{frame}
\titlepage % Print the title page as the first slide
\end{frame}

\begin{frame}
\frametitle{Overview} % Table of contents slide, comment this block out to remove it
%\tableofcontents % Throughout your presentation, if you choose to use \section{} and \subsection{} commands, these will automatically be printed on this slide as an overview of your presentation
\begin{itemize}
    \item Motivating Graphs

    \item Quick Literature Review

    \item Endgame Model

    \item Current Model

    \item Steady State Calculations

    \item Dynamic Preview

\end{itemize}
\end{frame}

%----------------------------------------------------------------------------------------
%	PRESENTATION SLIDES
%----------------------------------------------------------------------------------------


%------------------------------------------------
\section{Motivation} 
%------------------------------------------------
\begin{frame}
    \frametitle{US Budget}
\begin{figure}
	\centering
\includegraphics[scale=0.5]{chart.png}
	\label{V1}
\end{figure}

\end{frame}




\begin{frame}
    \frametitle{Japan Budget}
\begin{figure}
	\centering
\includegraphics[scale=0.5]{japan_budget.jpg}
	\label{V2}
\end{figure}

\end{frame}


\begin{frame}
    \frametitle{Government Debt and Fertility}
\begin{figure}
	\centering
\includegraphics[scale=0.5]{Debt_Fert.png}
	\label{V3}
\end{figure}

\end{frame}



\begin{frame}
    \frametitle{Current Account and Fertility}
\begin{figure}
	\centering
\includegraphics[scale=0.5]{CA_Fert.png}
	\label{V4}
\end{figure}

\end{frame}

\begin{frame}
    \frametitle{Underlying Question}

    \begin{itemize}
        \item What is the impact of a Pay-As-You-Go Pension System and a shrinking population on international financial allocation?
        \item Based on my data search, I hypothesize that countries with a PAYG pension and shrinking populations will have current account surpluses.
        \item I will begin this exploration by building a model. This will be an overlapping generations model (OLG).

    \end{itemize}


\end{frame}

%------------------------------------------------
\section{Lit Review} 
%------------------------------------------------
\begin{frame}
    \frametitle{A Brief Review}

    \begin{itemize}
        \item \textbf{OLG in International Trade/Finance}
            \begin{itemize}
                \item Stavely-O'Carroll and Stavely-O'Carroll (2017): Comparing two countries with and without PAYG system.
                \item Eugeni (2015): Differences in PAYG execution lead to impact on current accounts
                \item Sayan (2005): Two Countries growing at different (but constant) rates
            \end{itemize}

        \item \textbf{OLG/Pension and Saving}
            \begin{itemize}
                \item Samwick (2000): Pension system's impact on savings, Empirical evidence that it does distort savings decision.

            \end{itemize}

    \end{itemize}

\end{frame}

%------------------------------------------------
\section{Ideal Model} 
%------------------------------------------------
\begin{frame}
    \frametitle{Unique Features}
I am working to build a two good, two country OLG Trade Model with the following features:
    \begin{itemize}
        \item Both countries have a Pay-As-You-Go (PAYG) pension system and exhibit population growth.
        \item However, one will have stochastic population growth and the other will grow at a constant rate.
        \item The governments will be permitted to borrow to finance pensions
        \item Households can purchase good from either firms. 
        \item The intent is to examine the differences savings behavior between these two countries.

    \end{itemize}


\end{frame}



\begin{frame}
    \frametitle{Ideally..}
\frametitle{Full Model Diagram}
\begin{figure}
	\centering
	\includegraphics[scale=0.35]{Full_Model.png}
	\label{V5}
\end{figure}
\end{frame}

%------------------------------------------------
\section{Current Model} 
%------------------------------------------------
\begin{frame}
    \frametitle{Starting from the Ground Up (Households)}
    First, I'll focus on the stochastic population mechanic before adding the other features. The Household's problem is:
    \[\max_{s_t,c_t^y,c_{t+1}^o} u(c_t^y)+\beta \mathds{E}_t u(c_{t+1}^o)\]
    \[c_t^y = w_t - x_t - s_t\]
    \[c_{t+1}^o = p_{t+1} + (1+r_{t+1}) s_t\]
    Where the population grows:
    \[N_t = (1+g_t e^{z_t}) N_{t-1}\]
    Where $z_t = \rho z_{t-1} + \epsilon_t \text{   }\epsilon_t \sim N(0,\sigma_t^2)$


\end{frame}

\begin{frame}
    \frametitle{Households Continued}
    Households also have CRRA Preferences:

    \begin{equation}
        u(c)=\frac{c^{1-\gamma}}{1-\gamma}
    \end{equation}

    We normalize all of the variables in the following way:

     \[\hat{\theta}=\frac{\theta}{N}\]


\end{frame}

\begin{frame}
    \frametitle{Firm and Government}
    Pensions must equal contributions
    \[N_t x_t = N_{t-1} p_t \implies p_t = (1+e^{z_t}g_t)x_t\]
    Then the firm's (per capita) problem is:
    \[\max_{k_t} k_t^\alpha -w_t -r_t k_t\]
    With the standard factor prices:
    \[r_t = \alpha k_t^{\alpha-1}\]
    \[w_t = (1-\alpha)k_t^{\alpha}\]

\end{frame}

\begin{frame}
    \frametitle{Equilibrium Conditions}
    Given factor prices $(w_t,r_t)$, $x_t$ and $k_t$ %satisfy (with ):

    \[(c_t^y)^{-\gamma}=\beta \mathds{E}_t(1+r_{t+1})(c_{t+1}^o)^{-\gamma}\]
    \[k_t^\alpha = c_t^y + c_{t}^o + k_t +x_t\]
    Using the constraints defined before.

\end{frame}

%------------------------------------------------
\section{Steady State} 
%------------------------------------------------
\begin{frame}
    \frametitle{"Calibrations"}
    \begin{table}
\begin{tabular}{lr}
\toprule
{Parameter} & Value \\
\midrule
$g$  &  0.03 \\
$\beta$  &  0.95 \\
$\alpha$  &  0.35\\
$\delta$  &  0.04\\
$\gamma$  &  3 \\
$\rho$ & 0.8 \\
$\sigma_e$& 0.03\\
\bottomrule
\end{tabular}
\end{table}

\end{frame}

\begin{frame}
    \frametitle{"Results" unconstrained}
\begin{table}[]
\centering
\label{my-label}
\begin{tabular}{|l|l|}
\hline
Steady State  & Value   \\ \hline
$k_{ss}$           & 0.00155394127258     \\ \hline
$x_{ss}$           & 0.0387485406436 \\ \hline
$r_{ss}$           & 23.4226243488    \\ \hline
$w_{ss}$           & 0.0675951392773     \\ \hline
$c^y_{ss}$         & 0.0272926573611   \\ \hline
$c^o_{ss}$         & 0.0778623208233     \\ \hline
\end{tabular}
\end{table}

\end{frame}


%------------------------------------------------
\section{Next Steps\dots} 
%------------------------------------------------
\begin{frame}
    \frametitle{Next Stage}
    \begin{itemize}
        \item Expand to the dynamic model (VFI)
        \item Next, I'll incorporate bond markets.
        \item Then, I'll add trade.
    \end{itemize}

    You can follow my progress at: \\

    \url{https://github.com/jdclawson/JeffPhDRepo}


\end{frame}

%\begin{frame}
%\frametitle{Value Function, Original Grid}
%\begin{figure}
%	\centering
%	\includegraphics[scale=0.5]{V1.jpg}
%	\label{V1}
%\end{figure}
%\end{frame}


%----------------------------------------------------------------------------------------

\end{document}
